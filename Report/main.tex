\documentclass[conference]{IEEEtran}
\IEEEoverridecommandlockouts
% The preceding line is only needed to identify funding in the first footnote. If that is unneeded, please comment it out.
\usepackage{cite}
\usepackage{amsmath,amssymb,amsfonts}
\usepackage{algorithmic}
\usepackage{graphicx}
\usepackage{textcomp}
\usepackage{xcolor}
\def\BibTeX{{\rm B\kern-.05em{\sc i\kern-.025em b}\kern-.08em
    T\kern-.1667em\lower.7ex\hbox{E}\kern-.125emX}}
\begin{document}

\title{The Observatory}

\author{\IEEEauthorblockN{1\textsuperscript{st} Fabio Plunser}
\and
\IEEEauthorblockN{2\textsuperscript{nd} Dominik Barbist}
}

\maketitle

\section{Introduction}
For this Project we need to implement a system that can detect unknown faces in a given environment(Iot device/ cameras). For that it will
use a camera to collect images, process them on the edge device and send the processed images to the cloud. In the cloud
we will use Amazon Rekognition to analyze the faces and compare them with a database(S3) of known faces. If an unknown face
is detected, we will send a signal to the edge device to trigger an alarm for the responding Iot device.
There are a couple of hurdles we need to overcome for this project. A high level of parallelism is needed to process the images
send from at least 5 cameras.\\
\\
\textbf{Main Steps:}
\begin{itemize}
    \item \textbf{Data Collection:} We need to collect images from the sensors and send over the edge device to the cloud.
    \item \textbf{Data Processing:} We need to do preceding on the images to extract the relevant information(faces) on the edge device.
    \item \textbf{Data Storage:} We need to store the processed images in the cloud(S3).
    \item \textbf{Data Analysis:} We need to analyze the faces in the cloud and cross reference them with faces in the database.
    \item \textbf{Signal Processing:} We need to send a signal to the edge device to notify the Iot device about an unknown face, and subsequently trigger an alarm.
    \item \textbf{User Interface:} We need to provide a user interface to the user to view the unknown faces(Maybe) and disarm the alarm.
\end{itemize}

\section{System architecture}
\begin{itemize}
    \item \textbf{Data Collection:} At the start we will emulate data collection by using WiseNet. Later we will use a real camera to collect data(Maybe).
    \item \textbf{Data Processing:} For the data processing we will use OpenCV, and for the face recognition we will use YOLO.
    \item \textbf{Data Storage:} We will use AWS S3 to store the processed images. Important to note that we will only store the processed images(faces) and not the raw images.
    \item \textbf{Data Analysis:} After the images are stored in S3, the edge device will send a signal to a lambda function which will trigger the face recognition process in Amazon Rekognition.
    \item \textbf{Signal Processing:} If an unknown face is detected, the lambda function will send a signal to the edge device to trigger an alarm.  
    \item \textbf{User Interface:} We will use a simple web interface to display the unknown faces and disarm the alarm.
\end{itemize}
Describe your system in detail, including a figure for your architectural diagram (IoT, Edge, Cloud layers, components developed and services used).

\begin{figure}[h!]
    \centering
    \includegraphics[width=1\linewidth]{images/architecture.excalidraw.png}
    \caption{Example of your architectural diagram.}
    \label{fig:enter-label}
\end{figure}

\section{Implementation details}
Provide details about the frameworks and resources used. Justify your decisions carefully.

\section{Evaluation}
TODO: For later after the implementation is done.
Evaluation of the response time and scalability (number of devices and traffic) to prove the correctness of your implementation. The more detailed the better. 

\end{document}
